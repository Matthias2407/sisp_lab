\documentclass[margin = 2cm]{article}

\usepackage{polski}
\usepackage{graphicx}
\usepackage{listings} %listingi kodu
\usepackage{array}
\usepackage{float}
\usepackage{amsmath}
\usepackage{multicol} %kolumny
\usepackage[margin=0.8in]{geometry}
\usepackage{gensymb} %stopnie (symbol)

\graphicspath{ {./images/} } %obrazy

\begin{document} %tytuł
\title{\textbf{Sprawozdanie z zajęć numer 2 \\Systemy i sieci przemysłowe}} % \\ robi nową linię
\author{Maciej Misiewicz\\215305 \and Oskar Zieliński\\215373 \and Dariusz Witek vel Witkowski\\215364 \and Szymon Panek\\215319}

\maketitle


\newpage
\tableofcontents
\newpage

\section{Cel ćwiczenia}
	Celem ćwiczenia było zapoznanie się z komunikacją za pomocą protokołu CAN na przykładzie połączenia z częścią składową robota hipermobilnego Wheeeler. Ćwiczenie obejmowało wysyłanie rozkazów oraz odbieranie informacji od lokalnego sterownika za pomocą ramek danych.
\section{Realizacja ćwiczenia}
	\subsection{Konfiguracja połączenia z robotem}
Konfiguracja połączenia z robotem ograniczała się do określenia dwóch parametrów - prędkości transmisji jako 1Mbit/s oraz formatu identyfikatora jako format extended.
		\begin{figure}[H]
			\centering
			\includegraphics[width=0.5\textwidth]{0}
			\caption{Okno nawiązywania połączenia}
		\end{figure}
Po udanym połączeniu zostały odebrane trzy ramki danych.
		\begin{figure}[H]
			\centering
			\includegraphics[width=0.9\textwidth]{0_1}
			\caption{Odebrane ramki danych}
		\end{figure}
		\begin{figure}[H]
			\centering
			\includegraphics[width=0.9\textwidth]{interfejs}
			\caption{Widok interfejsu w programie PCANView.}
		\end{figure}
		\begin{figure}[H]
			\centering
			\includegraphics[width=0.6\textwidth]{edytorinstrukcji}
			\caption{Kreator rozkazów}
		\end{figure}
	\subsection{Zadania}
		\subsubsection{Ustawienie cyklu pracy układu lokalnego 30ms}
\label{a}
Ustawienie cyklu pracy układu lokalnego odbywa się za pomocą 6 bajtu, podana w nim wartość w zakresie 2-20 (dec) zwiększających cykl co 10ms. Bajt 7 odpowiada za ustawienie zadanych wartości w instrukcji.  Ramka realizujaca zadanie wygląda następująco:

00000006h	00 00 00 00 00 00 03 40

		\begin{figure}[H]
			\centering
			\includegraphics[width=0.9\textwidth]{30ms}
		\end{figure}

		\subsubsection{Ustawienie cyklicznego odczytu danych z akcelerometru}
Ustawienie odczytu danych z akcelerometru odbywa się za pomocą 4 bajtu. Ustawiona wartość 07 jest składową odczytu osi X, Y i Z. Bajt 7 odpowiada za cykliczne odczytywanie pomiarów. Ramka realizujaca zadanie wygląda następująco:

00000006h	00 00 00 00 00 07 00 20
		\subsubsection{Ustawienie cyklicznego odczytu danych z czujników odległości}
Ustawienie odczytu danych z czujników odległości odbywa się za pomocą 6 bajtu. Ustawiona wartość F0 jest składową adresów 4 czujników odległości. Bajt 7 odpowiada za cykliczne odczytywanie pomiarów. 

00000006h	00 00 00 00 00 00 F0 20
		\subsubsection{Zmienić położenie serwa 1 i serwa 2}
Zakres roboczy obu serw mieści się między 0 a 300 (dec), naszym zadaniem było ustawienie 3 zadanych pozycji: minimalnej, środkowej oraz maksymalnej. Wysterowanie pojedynczego serwa odbywa się przy pomocy dwóch bajtów. Dla serwa 1 jest to bajt0 oraz bajt1, natomiast dla    serwa 2  bajt2 oraz bajt3.

max

00000006h	2C 01 2C 01 00 00 00 40
		\begin{figure}[H]
			\centering
			\includegraphics[width=0.9\textwidth]{max}
		\end{figure}
middle

00000006h	96 00 96 00 00 00 00 40
		\begin{figure}[H]
			\centering
			\includegraphics[width=0.9\textwidth]{middle}
		\end{figure}
min

00000006h	00 00 00 00 00 00 00 40
		\begin{figure}[H]
			\centering
			\includegraphics[width=0.9\textwidth]{min}
		\end{figure}
		\subsubsection{Zmienić cykl pracy układu lokalnego na 50ms}
Analogicznie do punktu \ref{a}, zmianie uległa jedynie zadana długość cyklu.

00000006h	00 00 00 00 00 00 05 40
		\begin{figure}[H]
			\centering
			\includegraphics[width=0.9\textwidth]{50ms}
		\end{figure}
\section{Wnioski i spostrzeżenia}
Oprogramowanie PCANView jest łatwym, czytelnym i intuicyjnym narzędziem. Bezpośrednio pokazuje komunikację za pomocą ramek, dzieląc je na przychodzące i wysyłane. Jest dobrym programem do testowania oraz podglądu transmisji protokołem CAN. Zaletą jest także możliwość wyboru sposobu wysyłania rozkazów. Możemy wysłać ramkę pojedynczo manualnie lub wysyłać cyklicznie w zadanym interwale. Program sygnalizuje status połączenia w protokole. 
\end{document}